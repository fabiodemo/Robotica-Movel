%%%%%%%%%%%%%%%%%%%%%%%%%%%%%%%%%%%%%%%%
% Beamer Presentation
% LaTeX Template
% Version 1.0 (10/11/12)
%
% This template has been downloaded from:
% http://www.LaTeXTemplates.com
%
% License:
% CC BY-NC-SA 3.0 (http://creativecommons.org/licenses/by-nc-sa/3.0/)
%
%%%%%%%%%%%%%%%%%%%%%%%%%%%%%%%%%%%%%%%%%

%----------------------------------------------------------------------------------------
%	PACKAGES AND THEMES
%----------------------------------------------------------------------------------------

\documentclass[xcolor=dvipsnames, aspectratio=169]{beamer}

\mode<presentation> {

% The Beamer class comes with a number of default slide themes
% which change the colors and layouts of slides. Below this is a list
% of all the themes, uncomment each in turn to see what they look like.

\usetheme{Madrid} %Hannover
% As well as themes, the Beamer class has a number of color themes
% for any slide theme. Uncomment each of these in turn to see how it
% changes the colors of your current slide theme.
\useoutertheme{infolines} % Alternatively: miniframes, infolines, split
\useinnertheme{circles}
\definecolor{UBCblue}{rgb}{0.04706, 0.13725, 0.26667} % UBC Blue (primary)
\usecolortheme[named=UBCblue]{structure}
}

\usepackage{graphicx} % Allows including images
\usepackage{booktabs} % Allows the use of \toprule, \midrule and \bottomrule in tables
\usepackage{textpos}
\usepackage{caption}
\usepackage[utf8]{inputenc}
\usepackage[brazilian]{babel}
\usepackage{csquotes}
\usepackage{listings}
\setbeamertemplate{caption}[numbered]
\usepackage[style=abnt]{biblatex}
\addbibresource{bibliography.bib}
\PassOptionsToPackage{useregional}{datetime2}
\usepackage{xcolor}
\usepackage{amsmath}


\definecolor{codegreen}{rgb}{0,0.6,0}
\definecolor{codegray}{rgb}{0.5,0.5,0.5}
\definecolor{codepurple}{rgb}{0.58,0,0.82}
\definecolor{backcolour}{rgb}{0.95,0.95,0.92}
\definecolor{string-color}{rgb}{0.3333, 0.5254, 0.345}

\lstdefinestyle{mystyle}{
    backgroundcolor=\color{backcolour},   
    commentstyle=\color{codegreen},
    keywordstyle=\color{string-color},
    keywordstyle=[2]{\color{codepurple}},
    keywordstyle=[3]{\color{magenta}},
    numberstyle=\tiny\color{codegray},
    stringstyle=\color{codepurple},
    basicstyle=\ttfamily\tiny,
    breakatwhitespace=false,         
    breaklines=true,                 
    captionpos=b,                    
    keepspaces=true,                 
    numbers=left,                    
    numbersep=5pt,                  
    showspaces=false,                
    showstringspaces=false,
    showtabs=false,                  
    tabsize=2,
    % language=Python, % Adiciona suporte para Python
    morekeywords={If, Then, Else, While, Do, For, Return, end, End, if}, % Adicione suas palavras-chave aqui
    morecomment=[l]{//}, % Define o estilo de comentário, ajuste conforme necessário
    morecomment=[s]{/*}{*/}, % Para comentários de bloco, se necessário
    morestring=[b]" % Para strings, se necessário
}

\lstset{style=mystyle}
\newcommand{\source}[1]{\vspace{-20pt} \caption*{ Fonte: {#1}} }
\usepackage{copyrightbox}


\makeatletter
% \beamer@nav@subsectionstyle{hide/hide/hide}
\addtobeamertemplate{sidebar left}{%
\hspace{0.5cm}\includegraphics[width=0.9cm, keepaspectratio]{figures/_brasao_ufsm_cor.png}
% \hspace{2.3cm}\includegraphics[width=0.8cm, keepaspectratio]{figures/brasao_ctism.png}
% \hspace{2.3cm}\includegraphics[width=1.5cm, keepaspectratio]{_logosbc.png}
% \hspace{2.3cm}\includegraphics[width=1.5cm, keepaspectratio]{_logoERRC.png}%
}{}


\setbeamertemplate{footline}
{
	\leavevmode%
	\hbox{%
	    % \hspace{0.5cm}\includegraphics[width=0.8cm, keepaspectratio]{figures/_brasao_ufsm_cor.png}
		\begin{beamercolorbox}[wd=.333333\paperwidth,ht=2.25ex,dp=1ex,right]{date in head/foot}%
			\usebeamerfont{date in head/foot}\insertshortdate{}\hspace*{2em}
			\insertframenumber{} / \inserttotalframenumber\hspace*{2ex} 
		\end{beamercolorbox}}%
		%\vskip0pt%
	}
\makeatother

%----------------------------------------------------------------------------------------
%	TITLE PAGE
%----------------------------------------------------------------------------------------

\title[An experimental system for incremental environment modelling by an autonomous mobile robot]{An experimental system for incremental environment modelling by an autonomous mobile robot} % The short title appears at the bottom of every slide, the full title is only on the title page

\author[FDR]{Fábio Demo da Rosa} % Your name
%\includegraphics[]{logositeredes.png}
\institute[UFSM] % Your institution as it will appear on the bottom of every slide, may be shorthand to save space
{
Universidade Federal de Santa Maria \\ % Your institution for the title page
Pós-Graduação em Ciência da Computação \\
Disciplina de Robótica Móvel\\
\medskip
\textit{faberdemo@gmail.com} % Your email address
}
\date{\today} % Date, can be changed to a custom date
\newcounter{saveenumi}
\newcommand{\seti}{\setcounter{saveenumi}{\value{enumi}}}
\newcommand{\conti}{\setcounter{enumi}{\value{saveenumi}}}

\resetcounteronoverlays{saveenumi}


\begin{document}

\begin{frame}
\titlepage % Print the title page as the first slide
\end{frame}

\begin{frame}
\frametitle{Visão Geral} %\includegraphics[]{logositeredes.png}} % Table of contents slide, comment this block out to remove it
\tableofcontents % Throughout your presentation, if you choose to use \section{} and \subsection{} commands, these will automatically be printed on this slide as an overview of your presentation
\end{frame}

%----------------------------------------------------------------------------------------
%	PRESENTATION SLIDES
%----------------------------------------------------------------------------------------

%------------------------------------------------
\section{Introdução}
%------------------------------------------------
\begin{frame}[fragile]
  \frametitle{Introdução}
  \begin{itemize}
    \item Mapeamento incremental por robôs autônomos.
    \item Desafio de inconsistências e imprecisões nos dados dos sensores.
    \item Processamento de imprecisões para modelagem consistente \cite{moutarlier2006experimental}.
  \end{itemize}
\end{frame}

%------------------------------------------------
\section{Importância do Mapeamento Incremental}
%------------------------------------------------
\begin{frame}
  \frametitle{Importância do Mapeamento Incremental}
  \begin{itemize}
      \item Operações-chave: 
      \begin{itemize}
        \item Percepção, reconhecimento, decisão e navegação.
      \end{itemize}
      \item Acúmulo de Erros: 
      \begin{itemize}
        \item Como os erros sensoriais se acumulam durante o processo de percepção-modelagem.
        \item Erros anteriores impactam diretamente nos cálculos atuais.
      \end{itemize}
  \end{itemize}
\end{frame}

%------------------------------------------------
\section{Abordagem Teórica}
%------------------------------------------------
\begin{frame}
  \frametitle{Abordagem Teórica}
  \begin{itemize}
      \item Uso de filtragem de Kalman para fusão de dados sensoriais estocásticos.
      \item Desafios dessa abordagem: 
      \begin{itemize}
        \item Instabilidade devido à natureza não linear do problema, especialmente sem uma boa calibração da variância de ruído do sensor.
      \end{itemize}
      \item Possível solução: 
      \begin{itemize}
        \item Formalismo que pode tratar a fusão de dados em contextos não lineares. 
        \item É proposto um sistema que pode ajustar a representação do estado do robô e do ambiente de forma incremental, considerando a incerteza e a correlação dos dados dos sensores \cite{moutarlier2006experimental}.
      \end{itemize}
  \end{itemize}
\end{frame}
  
%------------------------------------------------
\section{Modelagem Local e Global}
%------------------------------------------------
\begin{frame}[allowframebreaks, fragile]
  \frametitle{Modelagem Local e Global}
  \begin{itemize}
      \item Necessidade de Modelagem Local: 
      \begin{itemize}
        \item Conversão de dados brutos do sensor em um modelo local para correspondência com um modelo global.
      \end{itemize}
      \item Desafio com Dados Brutos:
      \begin{itemize}
        \item Dados brutos de sensores, como um localizador a laser, são geralmente pobres e instáveis para uso direto.
      \end{itemize}
      \item Segmentação e Extração de Características:
      \begin{itemize}
        \item Processos usados para criar uma representação de nível superior a partir dos dados brutos.
      \end{itemize}
      \begin{figure}
        \centering
        \copyrightbox[b]{\includegraphics[scale=0.30]{figures/2_local_segmentation.png}}%
        {Fonte: \cite{buniyamin2011simple}}
        \caption{Segmentação local.}
        \label{fig:1_range_sensor_obstacle}
      \end{figure}
  \end{itemize}
\end{frame}
  

%------------------------------------------------
\section{Ferramentas Matemáticas para Tratar Inexatidões dos Sensores}
%------------------------------------------------
  \begin{frame}[allowframebreaks, fragile]
  \frametitle{Ferramentas Matemáticas para Tratar Inexatidões dos Sensores}
  \begin{itemize}
      \item Uso de Sensores: 
      \begin{itemize}
        \item Diferentes tipos de sensores usados para mapeamento incremental.
      \end{itemize}
      \item Problemas com Erros de Movimento:
      \begin{itemize}
        \item Como os erros de movimento se acumulam com as imprecisões de percepção.
      \end{itemize}
      \item Abordagem Proposta:
      \begin{itemize}
        \item Uma abordagem baseada em métodos estocásticos para construir uma representação consistente do ambiente.
      \end{itemize}

      \begin{figure}
        \centering
        \copyrightbox[b]{\includegraphics[scale=0.30]{figures/1_Laser_Range_Finder_Raw_Data.png}}%
        {Fonte: \cite{buniyamin2011simple}}
        \caption{Dados brutos do sensor laser.}
        \label{fig:1_range_sensor_obstacle}
      \end{figure}
  \end{itemize}
\end{frame}
  

%------------------------------------------------
\section{Representação do ambiente e Posicionamento do Robô}
%------------------------------------------------
  \begin{frame}
  \frametitle{Representação Básica e Posicionamento do Robô}
  \begin{itemize}
      \item Comparação entre representação relacional e de localização.
      \begin{itemize}
        \item Representação Relacional: 
        \begin{itemize}
          \item Objeto são ligados através de transformações incertas, formando um gráfico onde as conexões espaciais são complexas de atualizar com novos dados. 
        \end{itemize}
        \item Representação de Localização:
        \begin{itemize}
          \item Usa um quadro de referência único/comum para todos os objetos, simplificando a atualização do modelo ambiental.
        \end{itemize}
      \end{itemize}
      \item Decisão de Design do artigo: 
      \begin{itemize}
        \item Uso de um quadro de referência único \cite{moutarlier2006experimental}.
        \item Tal quadro Facilita o uso de um vetor de estado único para representar todos os objetos, incluindo o robô.
      \end{itemize}
  \end{itemize}
\end{frame}
  

%------------------------------------------------
\section{Avaliação de Inexatidões e Calibração}
%------------------------------------------------
  \begin{frame}
  \frametitle{Avaliação de Inexatidões e Calibração}
  \begin{itemize}
      \item Avaliação da Deriva Odometria: 
      \begin{itemize}
        \item Estimativa da posição do robô com base na leitura de codificadores angulares nas rodas.
      \end{itemize}
      \item Calibração do Sensor Laser:
      \begin{itemize}
        \item Desafios e métodos usados para calibrar o sensor laser para melhor precisão.
      \end{itemize}
  \end{itemize}
\end{frame}
  

%------------------------------------------------
\section{Estratégias de Correspondência e Atualização do Modelo}
%------------------------------------------------
  \begin{frame}[allowframebreaks, fragile]
  \frametitle{Estratégias de Correspondência e Atualização do Modelo}
  \begin{itemize}
      \item Correspondência Heurística: 
      \begin{itemize}
        \item Correção da posição do robô baseada em heurísticas e correspondência de segmentos.
      \end{itemize}
      \item Correspondência Estocástica:
      \begin{itemize}
        \item  Uso de variações conhecidas para corresponder segmentos de percepção com recursos do modelo.
      \end{itemize}

      \begin{figure}
        \centering
        \copyrightbox[b]{\includegraphics[scale=0.28]{figures/5_heuristic_matchings.png}}%
        {Fonte: \cite{buniyamin2011simple}}
        \caption{Atualização dos Pontos de Extremidade.}
        \label{fig:5}
      \end{figure}

      \begin{figure}
        \centering
        % Figura 3
        \begin{minipage}{0.31\textwidth}
            \centering
            \copyrightbox[b]{\includegraphics[scale=0.25]{figures/3_last_model_and_Current_perceptions_superposition.png}}%
            {Fonte: \cite{buniyamin2011simple}}
            \caption{Superposição do Último Modelo e Percepções Atuais.}
            \label{fig:3}
        \end{minipage}
        \hfill % Espaço entre as figuras
        % Figura 6
        \begin{minipage}{0.31\textwidth}
            \centering
            \copyrightbox[b]{\includegraphics[scale=0.25]{figures/6_endpoints_updating.png}}%
            {Fonte: \cite{buniyamin2011simple}}
            \caption{Correspondências Heurísticas.}
            \label{fig:6}
        \end{minipage}
        \hfill % Espaço entre as figuras
        % Figura 7
        \begin{minipage}{0.31\textwidth}
            \centering
            \copyrightbox[b]{\includegraphics[scale=0.25]{figures/7_global_approach_instability.png}}%
            {Fonte: \cite{buniyamin2011simple}}
            \caption{Atualização dos Pontos de Extremidade.}
            \label{fig:7}
        \end{minipage}
    \end{figure}

  \end{itemize}
\end{frame}
  

%------------------------------------------------
\section{Considerações finais}
%------------------------------------------------
  \begin{frame}[allowframebreaks, fragile]
  \frametitle{Considerações finais}
  \begin{itemize}
      \item Capacidade do Sistema:
      \begin{itemize}
        \item Demonstração da capacidade do sistema em lidar com imprecisões e construir um modelo ambiental consistente.
      \end{itemize}
      \item Desafios Futuros:
      \begin{itemize}
        \item Melhorias necessárias no modelo de imprecisões dos sensores, especialmente para odometria.
      \end{itemize}
      \begin{figure}
        \centering
        \copyrightbox[b]{\includegraphics[scale=0.30]{figures/8_actual_experiment_of_incremental_environment_modelling.png}}%
        {Fonte: \cite{buniyamin2011simple}}
        \caption{Experimento Atual de Modelagem Ambiental Incremental.}
        \label{fig:8}
      \end{figure}

      % \begin{figure}
      %   \centering
      %   \copyrightbox[b]{\includegraphics[scale=0.35]{figures/9_actual_experiment_of_incremental_environment_modelling.png}}%
      %   {Fonte: \cite{buniyamin2011simple}}
      %   \caption{Experimento Atual de Modelagem Ambiental Incremental.}
      %   \label{fig:9}
      % \end{figure}

  \end{itemize}
\end{frame}

%------------------------------------------------
%\section*{Referências}
%------------------------------------------------
\begin{frame}
    % \nocite{*}
    \printbibliography
\end{frame}


\begin{frame}
\titlepage % Print the title page as the first slide
\end{frame}

\end{document}