%%%%%%%%%%%%%%%%%%%%%%%%%%%%%%%%%%%%%%%%
% Beamer Presentation
% LaTeX Template
% Version 1.0 (10/11/12)
%
% This template has been downloaded from:
% http://www.LaTeXTemplates.com
%
% License:
% CC BY-NC-SA 3.0 (http://creativecommons.org/licenses/by-nc-sa/3.0/)
%
%%%%%%%%%%%%%%%%%%%%%%%%%%%%%%%%%%%%%%%%%

%----------------------------------------------------------------------------------------
%	PACKAGES AND THEMES
%----------------------------------------------------------------------------------------

\documentclass[xcolor=dvipsnames, aspectratio=169]{beamer}

\mode<presentation> {

% The Beamer class comes with a number of default slide themes
% which change the colors and layouts of slides. Below this is a list
% of all the themes, uncomment each in turn to see what they look like.

\usetheme{Madrid} %Hannover
% As well as themes, the Beamer class has a number of color themes
% for any slide theme. Uncomment each of these in turn to see how it
% changes the colors of your current slide theme.
\useoutertheme{infolines} % Alternatively: miniframes, infolines, split
\useinnertheme{circles}
\definecolor{UBCblue}{rgb}{0.04706, 0.13725, 0.26667} % UBC Blue (primary)
\usecolortheme[named=UBCblue]{structure}
}

\usepackage{graphicx} % Allows including images
\usepackage{booktabs} % Allows the use of \toprule, \midrule and \bottomrule in tables
\usepackage{textpos}
\usepackage{caption}
\usepackage[utf8]{inputenc}
\usepackage[brazilian]{babel}
\usepackage{csquotes}
\usepackage{listings}
\setbeamertemplate{caption}[numbered]
\usepackage[style=abnt]{biblatex}
\addbibresource{bibliography.bib}
\PassOptionsToPackage{useregional}{datetime2}
\usepackage{xcolor}
\usepackage{amsmath}


\definecolor{codegreen}{rgb}{0,0.6,0}
\definecolor{codegray}{rgb}{0.5,0.5,0.5}
\definecolor{codepurple}{rgb}{0.58,0,0.82}
\definecolor{backcolour}{rgb}{0.95,0.95,0.92}
\definecolor{string-color}{rgb}{0.3333, 0.5254, 0.345}

\lstdefinestyle{mystyle}{
    backgroundcolor=\color{backcolour},   
    commentstyle=\color{codegreen},
    keywordstyle=\color{string-color},
    keywordstyle=[2]{\color{codepurple}},
    keywordstyle=[3]{\color{magenta}},
    numberstyle=\tiny\color{codegray},
    stringstyle=\color{codepurple},
    basicstyle=\ttfamily\tiny,
    breakatwhitespace=false,         
    breaklines=true,                 
    captionpos=b,                    
    keepspaces=true,                 
    numbers=left,                    
    numbersep=5pt,                  
    showspaces=false,                
    showstringspaces=false,
    showtabs=false,                  
    tabsize=2,
    % language=Python, % Adiciona suporte para Python
    morekeywords={If, Then, Else, While, Do, For, Return, end, End, if}, % Adicione suas palavras-chave aqui
    morecomment=[l]{//}, % Define o estilo de comentário, ajuste conforme necessário
    morecomment=[s]{/*}{*/}, % Para comentários de bloco, se necessário
    morestring=[b]" % Para strings, se necessário
}

\lstset{style=mystyle}
\newcommand{\source}[1]{\vspace{-20pt} \caption*{ Fonte: {#1}} }
\usepackage{copyrightbox}


\makeatletter
% \beamer@nav@subsectionstyle{hide/hide/hide}
\addtobeamertemplate{sidebar left}{%
\hspace{0.5cm}\includegraphics[width=0.9cm, keepaspectratio]{figures/_brasao_ufsm_cor.png}
% \hspace{2.3cm}\includegraphics[width=0.8cm, keepaspectratio]{figures/brasao_ctism.png}
% \hspace{2.3cm}\includegraphics[width=1.5cm, keepaspectratio]{_logosbc.png}
% \hspace{2.3cm}\includegraphics[width=1.5cm, keepaspectratio]{_logoERRC.png}%
}{}


\setbeamertemplate{footline}
{
	\leavevmode%
	\hbox{%
	    % \hspace{0.5cm}\includegraphics[width=0.8cm, keepaspectratio]{figures/_brasao_ufsm_cor.png}
		\begin{beamercolorbox}[wd=.333333\paperwidth,ht=2.25ex,dp=1ex,right]{date in head/foot}%
			\usebeamerfont{date in head/foot}\insertshortdate{}\hspace*{2em}
			\insertframenumber{} / \inserttotalframenumber\hspace*{2ex} 
		\end{beamercolorbox}}%
		%\vskip0pt%
	}
\makeatother

%----------------------------------------------------------------------------------------
%	TITLE PAGE
%----------------------------------------------------------------------------------------

\title[A Simple Local Path Planning Algorithm for Autonomous Mobile Robots]{A Simple Local Path Planning Algorithm for Autonomous Mobile Robots} % The short title appears at the bottom of every slide, the full title is only on the title page

\author[FDR]{Fábio Demo da Rosa} % Your name
%\includegraphics[]{logositeredes.png}
\institute[UFSM] % Your institution as it will appear on the bottom of every slide, may be shorthand to save space
{
Universidade Federal de Santa Maria \\ % Your institution for the title page
Pós-Graduação em Ciência da Computação \\
Disciplina de Robótica Móvel\\
\medskip
\textit{faberdemo@gmail.com} % Your email address
}
\date{\today} % Date, can be changed to a custom date
\newcounter{saveenumi}
\newcommand{\seti}{\setcounter{saveenumi}{\value{enumi}}}
\newcommand{\conti}{\setcounter{enumi}{\value{saveenumi}}}

\resetcounteronoverlays{saveenumi}


\begin{document}

\begin{frame}
\titlepage % Print the title page as the first slide
\end{frame}

\begin{frame}
\frametitle{Visão Geral} %\includegraphics[]{logositeredes.png}} % Table of contents slide, comment this block out to remove it
\tableofcontents % Throughout your presentation, if you choose to use \section{} and \subsection{} commands, these will automatically be printed on this slide as an overview of your presentation
\end{frame}

%----------------------------------------------------------------------------------------
%	PRESENTATION SLIDES
%----------------------------------------------------------------------------------------

%------------------------------------------------
\section{Introdução}
%------------------------------------------------
\begin{frame}[fragile]
  \frametitle{Introdução}
  \begin{itemize}
    \item O planejamento de trajetória/rota/caminho é um elemento crucial para robôs móveis.
    \item É necessário determinar rotas, para passar ou chegar até pontos específicos do ambiente.
    \item As abordagens para planejar a trajetória são de acordo com o ambiente, o tipo de sensor, as capacidades do robô, entre outros.
    \begin{itemize}
      \item tais abordagens estão gradualmente buscando um melhor desempenho em termos de tempo, distância, custo e complexidade \cite{buniyamin2011simple}.
    \end{itemize}
  \end{itemize}
\end{frame}

%------------------------------------------------
\section{Abordagens de Planejamento de Trajetória}
%------------------------------------------------
\begin{frame}[fragile]
  \frametitle{Abordagens de Planejamento de Trajetória}
  \begin{itemize}
    \item Deseja-se encontrar caminhos adequados para um robô com geometria específica.
    \item Objetiva-se alcançar uma posição e orientação final a partir de uma inicial.
    \item O problema de navegação do robô móvel pode ser dividido em três subtarefas, de acordo com \cite{buniyamin2011simple}: 
    \begin{itemize}
      \item \textbf{Mapeamento e Modelagem do Ambiente:} o robô deve ser capaz de construir um mapa do ambiente e modelar o ambiente;
      \item \textbf{Planejamento de Trajetória:} o robô deve ser capaz de planejar um caminho para alcançar o objetivo;
      \item \textbf{Travessia da Trajetória:} o robô deve ser capaz de seguir o caminho planejado e evitar colisões com obstáculos.
    \end{itemize}
    \item Tipos de ambiente: estático (sem objetos móveis) e dinâmico (com objetos móveis).
    \item Abordagens: planejamento local e global.
  \end{itemize}
  
\end{frame}

% %------------------------------------------------
% \section{Evolução da Modelagem de Ambiente do Robô}
% %------------------------------------------------
% \begin{frame}[fragile]
%   \frametitle{Evolução da Modelagem de Ambiente do Robô}
%   \begin{itemize}
%     \item \textbf{Modelagem de Ambiente:} 
%     \begin{itemize}
%         \item Década de 1980: Introdução de Cones Generalizados (Generalized Cones) por Rodney Brooks. Funciona em ambientes simples, mas limitado em complexidade.
%         \begin{itemize}
%           \item Abordagem de Mapa Rodoviário (Roadmap), incluindo Grafos de Visibilidade (Visibility Graph) e Diagramas de Voronoi. Efetivos em ambientes simples, mas complexos e demorados na criação.
%         \end{itemize}
%         \item Década de 1990: Abordagem de Decomposição Celular (Cell Decomposition) torna-se popular, adequada para ambientes estáticos e dinâmicos, fácil implementação e atualização, mas inicialmente lenta em computadores antigos.
%         \item Século XXI: Introdução de Quadtree e Framed Quadtree, e Grafos MAKLINK, para aumentar a precisão de caminhos encontrados.
%     \end{itemize}
%   \end{itemize}
  
% \end{frame}


%------------------------------------------------
\section{Planejamento de Trajetória (Local e Global)}
%------------------------------------------------
\begin{frame}[fragile]
  \frametitle{Planejamento de Trajetória (Local e Global)}
  \begin{itemize}
    \item \textbf{Planejamento de Trajetória Global:} 
    \begin{itemize}
        \item Exige informação prévia do ambiente.
        \item Planejamento completo do trajeto antes do movimento do robô.
        \item Abordagens incluem Grafos de Visibilidade, Diagramas de Voronoi, Decomposição Celular, e métodos modernos como Algoritmo Genético, Redes Neurais e Otimização de Colônia de Formigas.
    \end{itemize}
    \item \textbf{Planejamento de Trajetória Local:}
    \begin{itemize}
      \item Essencial para robôs em ambientes dinâmicos, foca na evasão de obstáculos usando sensores.
      \item Robôs seguem a rota mais direta, alterando-a ao encontrar obstáculos.
      \item \textbf{Algoritmo PointBug:} Evita o perímetro dos obstáculos, diferenciando-se dos algoritmos Bug tradicionais.
      \begin{itemize}
        \item \textbf{Implementação do PointBug:} Utiliza sensores de curto alcance e sistemas de navegação para decisões de trajetória.
      \end{itemize}
    \end{itemize}
  \end{itemize}

\end{frame}

% %------------------------------------------------
% \section{Algoritmos Bug}
% %------------------------------------------------
% \begin{frame}[fragile]
%   \frametitle{Algoritmos Bug}
%   \begin{itemize}
%     \item Algoritmos Bug são métodos de navegação para robôs móveis em planejamento de trajetória local, com sensores mínimos e algoritmos simples.
%     \item Bug1 é cauteloso e cobre mais do que o perímetro total do obstáculo, enquanto Bug2 é menos eficiente em alguns casos, mas tem cobertura menor.
%     \item \textbf{Variações:} 
%     \begin{itemize}
%       \item Bug1, Bug2, DistBug, VisBug e TangentBug. Outras variações são Alg1, Alg2, Class1, Rev/Rev2, OneBug e LeaveBug.
%     \end{itemize}
%     \item \textbf{Evolução e Melhorias:} 
%     \begin{itemize}
%       \item Caminhos mais curtos, menor tempo, algoritmos mais simples e melhor desempenho.
%     \end{itemize}
%     \item \textbf{Aplicação em Planejamento de Trajetória Local:} 
%     \begin{itemize}
%       \item Utilizados para navegação em ambientes desconhecidos ou dinâmicos, adaptando-se a mudanças e obstáculos.
%     \end{itemize}
%   \end{itemize}

% \end{frame}

%------------------------------------------------
\section{Algoritmos PointBug}
%------------------------------------------------
\begin{frame}[allowframebreaks, fragile]
  \frametitle{Algoritmos PointBug}
  \begin{itemize}
    \item O PointBug é um algoritmo de navegação recentemente desenvolvido, para robôs em ambientes planares desconhecidos com obstáculos estacionários.
    \item Utiliza um sensor de alcance para detectar mudanças súbitas na distância até o obstáculo mais próximo (detecção de pontos).
    \begin{itemize}
      \item Mudanças súbitas são identificadas quando a distância varia significativamente em um curto intervalo de tempo.
    \end{itemize}
    \item Eficaz para resolver o problema de mínimos locais em ambientes desconhecidos, identificando pontos súbitos de maneira confiável.
    \item \textbf{Determinação do Próximo Ponto:} O próximo ponto de movimento é determinado pela saída do sensor de alcance, baseando-se na variação ($\Delta{d}$) da distância detectada.
    \item \textbf{Funcionamento do Algoritmo:} O robô inicialmente se direciona ao ponto-alvo, rotacionando para localizar um ponto súbito e mover-se em sua direção. Esse processo se repete até o robô atingir o ponto-alvo.
\begin{lstlisting}
While Not Target
  If robot rotation <= 360
    Robot rotates right of left according to position of dmin
  If sudden point
      If 180 degree rotation
        Ignore reading /* to avoid robot return to previous point */
      Else
        Get distance from current sudden point to next sudden point
        Get angle of robot rotation
        Move to new point according to distance and rotation angle
        Record New dmin value
        Reset rotation
      End if
    End if
  Else
    Robot Stop /* No sudden point and exit loop */
  End if
While end
Robot Stop /* Robot successfully reaches target */
\end{lstlisting}
  \end{itemize}
  % "A Figura 3 ilustra um sensor de distância escaneando um obstáculo em forma de pentágono, identificado pelos pontos de A a E. Acompanhando a figura, há um gráfico que representa a distância medida pelo sensor em centímetros, também de A a E. A linha no ponto C é perpendicular à superfície do obstáculo, marcando a menor distância detectada até ele. Observa-se que o valor da distância aumenta constantemente de C para B e de C para D. Entretanto, do ponto B para A, o gráfico mostra um aumento repentino na distância, quase dobrando o valor, e de D para E, o aumento da distância ocorre de alguns centímetros para o infinito. Os pontos A e E são identificados como pontos súbitos, sendo considerados pontos de referência para a próxima movimentação do robô. "
  \begin{figure}
    \centering
    \copyrightbox[b]{\includegraphics[scale=0.35]{figures/1_Range_sensor_is_detecting_an_obstacle_from_left_to_right_and_right_to_left.png}}%
    {Fonte: \cite{buniyamin2011simple}}
    \caption{O sensor de alcance está detectando um obstáculo da esquerda para a direita e da direita para a esquerda.}
    \label{fig:1_range_sensor_obstacle}
  \end{figure}
  % A Figura 4 complementa ao demonstrar como os pontos súbitos são detectados em obstáculos de diferentes formas.
  \begin{figure}
    \centering
    \copyrightbox[b]{\includegraphics[scale=0.30]{figures/2_Sudden_points_on_different_surfaces_detected_by_range_sensor.png}}%
    {Fonte: \cite{buniyamin2011simple}}
    \caption{Trajetória gerada pelo PointBug para resolver o problema de mínimos locais.}
    \label{fig:2_range_sensor_obstacle}
  \end{figure}
\end{frame}
  
%------------------------------------------------
\section{Simulação}
%------------------------------------------------
\begin{frame}[allowframebreaks, fragile]
  \frametitle{Simulação}
  \begin{itemize}
    \item A simulação do algoritmo de navegação ponto a ponto (PointBug) foi realizada utilizando ActionScript 2.0 no Adobe Flash CS3.
    \item Foram simulados três tipos de ambientes: um ambiente livre, um ambiente baseado em labirinto e um ambiente semelhante a um escritório.
  \end{itemize}
  % \begin{figure}
  %   \centering
  %   \copyrightbox[b]{\includegraphics[scale=0.30]{figures/3_Trajectory_generated_using_PointBug_algorithm_in_simple_maze_like_type_environment.png}}%
  %   {Fonte: \cite{buniyamin2011simple}}
  %   \caption{Trajetória gerada pelo PointBug para resolver o problema de mínimos locais.}
  %   \label{fig:2_range_sensor_obstacle}
  % \end{figure}

  \newpage
  \begin{figure}
    \centering
    \copyrightbox[b]{\includegraphics[scale=0.37]{figures/4_Trajectory_generated_using_Distbug,_TangenBug_and_PointBug_algorithm_in_Free_Environment.png}}%
    {Fonte: \cite{buniyamin2011simple}}
    \caption{Trajetória gerada pelos algoritmos em um ambiente livre.}
    \label{fig:4_trajectory_path}
  \end{figure}
  \newpage
  \begin{figure}
    \centering
    \copyrightbox[b]{\includegraphics[scale=0.35]{figures/5_Trajectory_generated_using_Distbug,_TangenBug_and_PointBugalgorithm_in_simple_Office_like_Environment.png}}%
    {Fonte: \cite{buniyamin2011simple}}
    \caption{Trajetória gerada pelos algoritmos em um ambiente simples (escritório).}
    \label{fig:5_trajectory_path}
  \end{figure}
\end{frame}


%------------------------------------------------
\section{Conclusões}
%------------------------------------------------
\begin{frame}[fragile]
  \frametitle{Conclusões}
  \begin{itemize}
    \item De acordo com o artigo \cite{buniyamin2011simple}, tratou-se sobre:
    \begin{itemize}
      \item Introdução do Algoritmo de Planejamento de Trajetória (PointBug), que possui mínima necessidade de informações prévias do ambiente.
      \item Provou-se a eficiência do PointBug em ambientes dinâmicos, utilizando informações em tempo real de sensores de alcance.
      \item Desempenho é diretamente influenciado pelo número de pontos súbitos detectados.
      \begin{itemize}
        \item Menos pontos indicam maior eficiência.
      \end{itemize}
      \item Efetividade dependente diretamente da forma dos obstáculos.
      \begin{itemize}
        \item Obstáculos circulares geram menos pontos súbitos.
      \end{itemize} 
    \end{itemize}

  \end{itemize}
\end{frame}

%------------------------------------------------
%\section*{Referências}
%------------------------------------------------
\begin{frame}
    % \nocite{*}
    \printbibliography
\end{frame}


\begin{frame}
\titlepage % Print the title page as the first slide
\end{frame}

\end{document}