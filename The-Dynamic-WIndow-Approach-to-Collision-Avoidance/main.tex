\documentclass[12pt]{article}

% This is the preamble, load any packages you're going to use here
\usepackage{physics} % provides lots of nice features and commands often used in physics, it also loads some other packages (like AMSmath)
\usepackage{siunitx} % typesets numbers with units very nicely
\usepackage{enumerate} % allows us to customize our lists



\begin{document}

\title{Interference \& Diffraction Report}
\author{A.~Student, Another Student, Last Student}
\date{\today}

\maketitle

\begin{abstract}
	Eventually we will learn to write abstracts for our lab reports, but not today, please delete this abstract.
\end{abstract}

\section{Section Name}
	It is helpful to label the different sections of your report.
    Please choose names that make sense.  You may number the sections, like in this example or not (then put an asterisk after the section command, e.g.~\verb+\section*+). 
    
    
\section{Analysis}
	Here you discuss your observations and results
	
    \subsection{Double Slit Interference}
		The subsection command let's you further divide your sections up.  Comment on your observations and results here
        
        \subsubsection{Varying Parameters}
        	For longer documents, you can even use subsubsections, 
            it's probably overkill for this analysis.
            
            \paragraph{Still More labels}
            	You can also label paragraphs
                
                \subparagraph{Subparagraphs}
                	And even subparagraphs.  Fortunately, we won't be 
                    writing such long documents in this course!
       
       \subsubsection{When to Make New Sections}
        	As a general rule of thumb, don't make smaller sections, subsections, etc, unless there are at least two of them at that level (just like with lists).
	
    \subsection{Single Slit}
		More discussion here
        
\section{Data}
	Please don't include any data tables in your \LaTeX\, write up.  Making tables in \LaTeX\, is very boring, although there are programs to convert your Excel file to \LaTeX\, form!  We'll worry about that later.



\end{document}