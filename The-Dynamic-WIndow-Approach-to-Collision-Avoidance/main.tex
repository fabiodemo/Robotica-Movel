\documentclass[12pt, a4paper]{article}
\usepackage[top=3cm, left=3cm, right=2cm, bottom=2cm]{geometry}

% This is the preamble, load any packages you're going to use here
\usepackage{physics} % provides lots of nice features and commands often used in physics, it also loads some other packages (like AMSmath)
\usepackage{siunitx} % typesets numbers with units very nicely
\usepackage{enumerate} % allows us to customize our lists
\renewcommand{\refname}{Referências}


\begin{document}

\title{Resumo do Artigo The Dynamic Window Approach to Collision Avoidance}
\author{Fábio Demo da Rosa}
\date{\today}

\maketitle

\section*{Resumo}
	O artigo tem como foco principal a prevenção de colisão em robôs móveis em ambientes perigosos ou com grande densidade populacional.
	A técnica principal empregada no artigo, chamada de \textit{Dynamic Windows Approach}, considera um curto intervalo de tempo ao computar o próximo cálculo de direção, evitando a complexidade de um problema de planejamento de rota convencional.

	O robô possui um limite de aceleração (para que consiga parar rapidamente e com segurança), além de outra restrição de velocidade imposta: o robô só considera velocidades que podem ser alcançadas durante o próximo intervalo de tempo. 
	Tais velocidades formam a \textit{Dynamic Window}, que é a combinação das velocidades atuais do robô no espaço de velocidades possíveis.

	As velocidades admissíveis para o veiculo estão dentro da \textit{Dynamic Window}, combinando velocidade translacional e rotacional, sendo a velocidade escolhida ao maximizar a \textit{objective function}. 
	
	A \textit{Objetive Function} inclui a medição do progresso em direção a uma localização desejada, a velocidade ao robô avançar e a distância do próximo obstáculo na trajetória.
	Ao combinar esses itens, o robô troca seu desejo de se mover rapidamente até o objetivo e o seu desejo de contonar os obstáculos.

	O artigo, próximo ao fim, detalha a implementação, testes e os resultados experimentais usando o robô RHINO, que utiliza vários sensores e uma câmera estéreo. A abordagem da janela dinâmica foi testada neste robô, com bons resultados observados em sua capacidade de navegar com segurança e eficiência.

	Por fim, conclui-se que a \textit{Dynamic Window} é uma interessante abordagem para determinar o comportamento do robô, principalmente em cenários onde a navegação seja mais complicada. Com as abordagens e experimentos dos autores, observou-se: a decisão do robô seguir determinado caminho dependia de sua velocidade atual e possíveis acelerações, bem como a capacidade do robô se locomover-se por corredores (mesmo com obstáculos).

	\begin{thebibliography}{2}

		\bibitem{dynamic_window} Dieter, F. Wolfram, B. , Thrunyz, S. 1997. The dynamic window approach to collision avoidance. IEEE Robotics \& Automation Magazine
		
		\end{thebibliography}
		
	\end{document} 

\end{document}