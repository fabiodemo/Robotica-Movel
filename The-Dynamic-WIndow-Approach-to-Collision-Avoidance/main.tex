\documentclass[12pt]{article}

% This is the preamble, load any packages you're going to use here
\usepackage{physics} % provides lots of nice features and commands often used in physics, it also loads some other packages (like AMSmath)
\usepackage{siunitx} % typesets numbers with units very nicely
\usepackage{enumerate} % allows us to customize our lists



\begin{document}

\title{Resumo do Artigo The Dynamic Window Approach to Collision Avoidance}
\author{Fábio Demo da Rosa}
\date{\today}

\maketitle

\section{Resumo}
	O artigo tem como foco principal a prevenção de colisão em robôs móveis em ambientes perigosos ou com grande densidade populacional. Utilizando uma técnica de prevenção de colisão reativa, ligando com impeditivos da limitação de velocidade e aceleração.
	A técnica principal empregada no artigo, chamada de \textit{Dynamic Windows Approach}, considera periodicamete um curto intervalo intervalo de tempo ao computar o próximo cálculo de direção, evitando a complexidade de um problema de planejamento de rota convencional.


\end{document}