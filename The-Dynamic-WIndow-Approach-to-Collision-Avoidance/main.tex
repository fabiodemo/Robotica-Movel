\documentclass[12pt, a4paper]{article}
\usepackage[top=3cm, left=3cm, right=2cm, bottom=2cm]{geometry}

% This is the preamble, load any packages you're going to use here
\usepackage{physics} % provides lots of nice features and commands often used in physics, it also loads some other packages (like AMSmath)
\usepackage{siunitx} % typesets numbers with units very nicely
\usepackage{enumerate} % allows us to customize our lists



\begin{document}

\title{Resumo do Artigo The Dynamic Window Approach to Collision Avoidance}
\author{Fábio Demo da Rosa}
\date{\today}

\maketitle

\section{Resumo}
	O artigo tem como foco principal a prevenção de colisão em robôs móveis em ambientes perigosos ou com grande densidade populacional. Utilizando uma técnica de prevenção de colisão reativa, ligando com impeditivos da limitação de velocidade e aceleração.
	A técnica principal empregada no artigo, chamada de \textit{Dynamic Windows Approach}, considera periodicamete um curto intervalo intervalo de tempo ao computar o próximo cálculo de direção, evitando a complexidade de um problema de planejamento de rota convencional.

	O robô por possuir limite de aceleração (para fazer com que o robô consiga parar rapidamente e com segurança),uma outra restrição de velocidade é imposta: o robô só considera velocidades que podem ser alcançadas durante o próximo intervalo de tempo. 
	Tais velocidades formam a \textit{Dynamic Window}, que é a combinação das velocidades atuais do robô no espaço de velocidades possíveis.

	As velocidades admissíveis para o veiculo estão dentro da \textit{Dynamic Window}, combinando velocidade translacional e rotacional, seno escolhida ao maximizar a \textit{objective function}. A \textit{Objetive Function} inclui a medição do progresso em direção a uma localização desejada, a velocidade e avanço do robô e a distância do próximo obstáculo na trajetória.
	Ao combinar esses itens, o robô se troca seu desejo de se mover rapidamente até o objetivo e o seu desejo de contonar os obstáculos.

	Em vez de avaliar o espaço de velocidade completo, é considerado apenas o espaço das velocidades admissíveis dentro da \textit{Dynamic Window}, levando em conta a dinâmica do robô. Um exemplo é dado mostrando como o comportamento do robô muda com diferentes velocidades.

	A dependência das acelerações também é debatida no artigo, mostrando que a janela dinâmica depende das acelerações dadas, e espaços de velocidade menores são considerados para acelerações menores.

	O artigo, próximo ao fim, detalha a implementação, testes e os resultados experimentais usando o robô RHINO, que utiliza vários sensores e uma câmera estéreo. A abordagem da janela dinâmica foi testada neste robô, com bons resultados observados em sua capacidade de navegar com segurança e eficiência.

\end{document}