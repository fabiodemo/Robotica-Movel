%%%%%%%%%%%%%%%%%%%%%%%%%%%%%%%%%%%%%%%%
% Beamer Presentation
% LaTeX Template
% Version 1.0 (10/11/12)
%
% This template has been downloaded from:
% http://www.LaTeXTemplates.com
%
% License:
% CC BY-NC-SA 3.0 (http://creativecommons.org/licenses/by-nc-sa/3.0/)
%
%%%%%%%%%%%%%%%%%%%%%%%%%%%%%%%%%%%%%%%%%

%----------------------------------------------------------------------------------------
%	PACKAGES AND THEMES
%----------------------------------------------------------------------------------------

\documentclass[xcolor=dvipsnames, aspectratio=169]{beamer}

\mode<presentation> {

% The Beamer class comes with a number of default slide themes
% which change the colors and layouts of slides. Below this is a list
% of all the themes, uncomment each in turn to see what they look like.

\usetheme{Madrid} %Hannover
% As well as themes, the Beamer class has a number of color themes
% for any slide theme. Uncomment each of these in turn to see how it
% changes the colors of your current slide theme.
\useoutertheme{infolines} % Alternatively: miniframes, infolines, split
\useinnertheme{circles}
\definecolor{UBCblue}{rgb}{0.04706, 0.13725, 0.26667} % UBC Blue (primary)
\usecolortheme[named=UBCblue]{structure}
}

\usepackage{graphicx} % Allows including images
\usepackage{booktabs} % Allows the use of \toprule, \midrule and \bottomrule in tables
\usepackage{textpos}
\usepackage{caption}
\usepackage[utf8]{inputenc}
\usepackage[brazilian]{babel}
\usepackage{csquotes}
\usepackage{listings}
\setbeamertemplate{caption}[numbered]
\usepackage[style=abnt]{biblatex}
\addbibresource{bibliography.bib}
\PassOptionsToPackage{useregional}{datetime2}
\usepackage{xcolor}


\definecolor{codegreen}{rgb}{0,0.6,0}
\definecolor{codegray}{rgb}{0.5,0.5,0.5}
\definecolor{codepurple}{rgb}{0.58,0,0.82}
\definecolor{backcolour}{rgb}{0.95,0.95,0.92}
\definecolor{string-color}{rgb}{0.3333, 0.5254, 0.345}

\lstdefinestyle{mystyle}{
    backgroundcolor=\color{backcolour},   
    commentstyle=\color{codegreen},
    keywordstyle=\color{string-color},
    keywordstyle=[2]{\color{codepurple}},
    keywordstyle=[3]{\color{magenta}},
    numberstyle=\tiny\color{codegray},
    stringstyle=\color{codepurple},
    basicstyle=\ttfamily\footnotesize,
    breakatwhitespace=false,         
    breaklines=true,                 
    captionpos=b,                    
    keepspaces=true,                 
    numbers=left,                    
    numbersep=5pt,                  
    showspaces=false,                
    showstringspaces=false,
    showtabs=false,                  
    tabsize=2,
    otherkeywords = {tf, Sequential, SimpleRNN, Dense, GRU, LSTM},
    morekeywords = [3]{keras},
}

\lstset{style=mystyle}
\newcommand{\source}[1]{\vspace{-20pt} \caption*{ Fonte: {#1}} }
\usepackage{copyrightbox}


\makeatletter
% \beamer@nav@subsectionstyle{hide/hide/hide}
\addtobeamertemplate{sidebar left}{%
\hspace{0.5cm}\includegraphics[width=0.9cm, keepaspectratio]{figures/_brasao_ufsm_cor.png}
% \hspace{2.3cm}\includegraphics[width=0.8cm, keepaspectratio]{figures/brasao_ctism.png}
% \hspace{2.3cm}\includegraphics[width=1.5cm, keepaspectratio]{_logosbc.png}
% \hspace{2.3cm}\includegraphics[width=1.5cm, keepaspectratio]{_logoERRC.png}%
}{}


\setbeamertemplate{footline}
{
	\leavevmode%
	\hbox{%
	    %\hspace{0.25cm}\includegraphics[width=2cm, keepaspectratio]{figures/brasao_ufsm_cor.png}
		\begin{beamercolorbox}[wd=.333333\paperwidth,ht=2.25ex,dp=1ex,right]{date in head/foot}%
			\usebeamerfont{date in head/foot}\insertshortdate{}\hspace*{2em}
			\insertframenumber{} / \inserttotalframenumber\hspace*{2ex} 
		\end{beamercolorbox}}%
		%\vskip0pt%
	}
\makeatother

%----------------------------------------------------------------------------------------
%	TITLE PAGE
%----------------------------------------------------------------------------------------

\title[]{Microwave Radar and Millimiter Radar} % The short title appears at the bottom of every slide, the full title is only on the title page

\author[]{Fábio Demo da Rosa} % Your name
%\includegraphics[]{logositeredes.png}
\institute[UFSM] % Your institution as it will appear on the bottom of every slide, may be shorthand to save space
{
Universidade Federal de Santa Maria \\ % Your institution for the title page
Pós-Graduação em Ciência da Computação \\
Disciplina de Robótica Móvel\\
\medskip
\textit{faberdemo@gmail.com} % Your email address
}
\date{25 de Agosto de 2023} % Date, can be changed to a custom date
\newcounter{saveenumi}
\newcommand{\seti}{\setcounter{saveenumi}{\value{enumi}}}
\newcommand{\conti}{\setcounter{enumi}{\value{saveenumi}}}

\resetcounteronoverlays{saveenumi}


\begin{document}

\begin{frame}
\titlepage % Print the title page as the first slide
\end{frame}

\begin{frame}
\frametitle{Visão Geral} %\includegraphics[]{logositeredes.png}} % Table of contents slide, comment this block out to remove it
\tableofcontents % Throughout your presentation, if you choose to use \section{} and \subsection{} commands, these will automatically be printed on this slide as an overview of your presentation
\end{frame}

%----------------------------------------------------------------------------------------
%	PRESENTATION SLIDES
%----------------------------------------------------------------------------------------

%------------------------------------------------
\section[Microwave Radar]{Microwave Radar} 
%------------------------------------------------
\begin{frame}[allowframebreaks, fragile]
\frametitle{Microwave Radar}
	\begin{itemize}
		\item A porção do espectro eletromagnético considerada uma frequência útil para radares práticos é entre 3 e 100 GHz;
		\item A maioria dos radares convencionais operam nas bandas L, S C ou X;
		\item A lista de letras (Figura) foi adotada como medida de segurança durante a Segunda Guerra Mundial, e foi mantida por conveniência;
		
		\begin{figure}
            \centering
            \copyrightbox[b]{\includegraphics[scale=0.45]{figures/1_MR_frequency_bands.png}}%
            {Fonte: \cite{everett1995sensors}}
            \caption{Bandas de frequência designadas para frequências de radares (IEEE Standard 521-1976).}
            \label{fig:curva_de_freq}
        \end{figure}

    \newpage
    \item O cálculo de distância é obtido por métodos TOF, CW phase Detection ou CW Frequency Modulation;
    \item \textit{Pulsed Systems} pode detectar alvos em distâncias de até centenas de quilômetros, dependendo na medida do tempo de propagação de uma onda propagada na velocidade da luz.
    \item \textit{Near-field measurements} (menos de 100 km) são mais difíceis para esse tipo de sistema;
    \begin{itemize}
        \item Pois sinais nítidos de curta duração são difíceis de se gerar para distâncias inferiores a um pé.
    \end{itemize} 
    \item Radares de onda contínua (CW) são efetivos para curtas distâncias.
    \begin{itemize}
        \item Pois \textit{phase-detection} ou \textit{frequency-shift} não são dependentes na velocidade da onda;
        \item Além de também serem adequadas para medir a velocidade de objetos em movimento por meio de métodos Doppler.
    \end{itemize}
	\end{itemize}
	
\end{frame}


    \subsection[Aplicações]{Aplicações} 
    \begin{frame}[allowframebreaks, fragile]
    \frametitle{Aplicações}
        \begin{itemize}
            \item Amplamente empregados em:
            \begin{itemize}
                \item Vigilância militar e comercial;
                \item Aplicações de navegação;
                \item Detecção de curto alcance (radar de alerta de controle para aeronaves);
                \item Indicadores de nível de tanques;
                \item Controles de tráfego e de velocidade de veículos;
                \item Sensores de movimento e detectores de presença;
                \item Forno micro-ondas.
            \end{itemize}
            \item As microondas são ideais para detecção de logo alcance, porque a resolução é geralmente boa, a atenuação dos feixes na atmosfera é minima;
            \begin{itemize}
                \item Operando em distâncias de alguns metros a algumas centenas de
                metros.
            \end{itemize}
            \item Equipamentos de transmissão, recepção e processamento da forma de onda estão amplamente disponíveis.
            \newpage
            \item De acordo com \cite{Agarwal_2021}, a radiação de micro-ondas é produzida por dispositivos de radar, antenas parabólicas, fornos de micro-ondas, entre outros;
            \begin{itemize}
                \item Sendo que essa pode afetar os seres humanos de formas diferentes, um exemplo pode ser o uso prolongado de telefone celular, conforme a Figura abaixo.
                \begin{figure}
                    \centering
                    \copyrightbox[b]{\includegraphics[scale=0.45]{figures/2_Microwave-radiation-effect-after-phone-call.jpg}}%
                    {Fonte: \cite{Agarwal_2021}}
                    % \caption{afetadas}
                    \label{fig:radiation}
                \end{figure}
            \end{itemize}
        \end{itemize}
    \end{frame}

    \subsection[Fatores de Performance]{Fatores de Performance} 
    \begin{frame}[allowframebreaks, fragile]
    \frametitle{Fatores de Performance}
        \begin{itemize}
            \item Aumentar o diâmetro do refletor resulta em uma melhoria na capacidade de alcance devido ao feixe de saída estar focalizado, e quanto mais larga a área da antena, maior a superfície de recepção/transmissão.
            \begin{itemize}
                \item Porém isso pode apresentar desvantagens em manipular um sistema mecânico com alta carga inercial.
            \end{itemize}
            		
		\begin{figure}
            \centering
            \copyrightbox[b]{\includegraphics[scale=0.7]{figures/3_MR_commom_configuration.png}}%
            {Fonte: \cite{everett1995sensors}}
            \caption{Configurações comuns das antenas de micro-ondas incluem: (A) prato refletor com ponto focal. (B) Antena tipo corneta. (C) matrizes bidimensionais de microfita}
            \label{fig:curva_de_freq}
        \end{figure}
        
        \item Muitas aplicações comerciais de curto alcance usam a antena tipo corneta para não lidar com os problemas citados;
        \item As configurações de antenas \textit{phased array} (Figura 9-9C) apresentam um arranjo de múltiplas antenas pequenas separadas por distâncias de alguns comprimentos de onda;
        \item Um dos fatores que atrapalha a performance significativamente é a atenuação atmosférica;
        \begin{itemize}
            \item 
        \end{itemize}
        \end{itemize}
    \end{frame}

%------------------------------------------------
\section[Millimeter-Wave Radar]{Millimeter-Wave Radar} 
%------------------------------------------------
\begin{frame}[allowframebreaks, fragile]
\frametitle{Millimeter-Wave Radar}
	\begin{itemize}
		\item O \textit{Frequency Modulated Continuous Wave Radar} (Radar de Onda contínua com Modulação de Frequência ou FMCW), é uma técnica alternativa ao \textit{Phase-Shift Measurement};
	\end{itemize}
\end{frame}


    \subsection[Introdução]{Aplicações} 
    \begin{frame}[allowframebreaks, fragile]
    \frametitle{Aplicações}
        \begin{itemize}
            \item Os usos mais comuns incluem:
            \begin{itemize}
                \item Sensoriamento ambiental;
                \item Radar de imagem com alta resolução;
                \item Espectroscopia;
                \item Equipamentos de telêmetro;
                \item Frenagem de automóveis.
            \end{itemize}
            \item Embora, o uso mais comum seja rastreamento e designação de alvos com fins militares \cite{everett1995sensors}.
            \item A estreita largura de feixe das transmissões de ondas milimétricas é altamente
            imune a problemas de reflexão do sol.
            \begin{itemize}
                \item Radares de busca de microondas de longo alcance
                e feixe largo para aquisição inicial e depois mudando para um radar de rastreamento
                milimétrico para controle do sistema de armas
            \end{itemize}
            \item Sistemas de ondas milimétricas de curto alcance e baixa potência parecem ser
            adequados para evitar colisões e necessidades de navegação de um robô móvel
            externos.
            \item a capacidade de usar antenas menores é uma característica
            dominante que influencia a seleção de ondas milimétricas em microondas. 
            \begin{itemize}
                \item As três plataformas mais diretamente afetadas: satélites, mísseis e mini-RPVs (\textit{Remotely Piloted Vehicles}).
            \end{itemize}
        \end{itemize}
    \end{frame}

    \subsection[Introdução]{Fatores de Performance} 
    \begin{frame}[allowframebreaks, fragile]
    \frametitle{Fatores de Performance}
        \begin{itemize}
            \item O
        \end{itemize}
    \end{frame}


%------------------------------------------------
%\section*{Referências}
%------------------------------------------------

\begin{frame}
    % \nocite{*}
    \printbibliography
\end{frame}


\begin{frame}
\titlepage % Print the title page as the first slide
\end{frame}

\end{document}